% !TEX TS-program = pdflatex
% !TEX encoding = UTF-8 Unicode

% This is a simple template for a LaTeX document using the "article" class.
% See "book", "report", "letter" for other types of document.

\documentclass[11pt]{article} % use larger type; default would be 10pt

\usepackage[portuguese]{babel}
\usepackage[utf8]{inputenc} % set input encoding (not needed with XeLaTeX)

%%% PAGE DIMENSIONS
\usepackage{geometry} % to change the page dimensions
\geometry{a4paper} % or letterpaper (US) or a5paper or....

\usepackage{graphicx} % support the \includegraphics command and options
\usepackage{booktabs} % for much better looking tables
\usepackage{array} % for better arrays (eg matrices) in maths
\usepackage{paralist} % very flexible & customisable lists (eg. enumerate/itemize, etc.)
\usepackage{verbatim} % adds environment for commenting out blocks of text & for better verbatim
\usepackage{subfig} % make it possible to include more than one captioned figure/table in a single float
\usepackage{listings}
\usepackage{amsmath}
\usepackage{mathtools}


%%% The "real" document content comes below...

\title{Dissertação de Mestrado em colaboração com a Accenture \\
\large Machine Learning e Processamento de Linguagens Naturais num Bot}
\author{Carlos António Senra Pereira\\
\small Orientador: Doutor Luís Filipe Ribeiro Pinto\\
\small Co-Orientador: Doutor José João Antunes Guimarães Dias de Almeida\\
\small Orientador na Accenture: Ricardo Perleques\\
\small Orientador na Accenture: Hugo André Portela}
%\date{} % Activate to display a given date or no date (if empty),
         % otherwise the current date is printed  

\begin{document}
\maketitle

\tableofcontents

\section{Resumo}

Resumo do Bot.

\section{Introdução}

\hspace{11pt} Com a capacidade de hardware cada vez mais evoluída e problemas quer da parte cientifica como problemas do quotidiano, Machine Learning tem sido vista como uma técnica possível para a resolição desses mesmos problemas.

O processamento de Linguagens Naturais é bastante complexo, e ainda mais no ponto de vista de uma máquina determinística. Por isso, utilizando técnicas de Machine Learning é possível construir um modelo capaz de analisar frases morfossintáticamente com bastante precisão.

Através desta análise, podemos tirar grande partido dela no interpretamento das mesmas e assim executar ações correspondentes às frases.

\subsection{Objetivos}

\begin{enumerate}
\item Construir uma aplicação capaz de analisar e reconhecer frases introduzidas por utilizadores;
\item Através de um modelo de Machine Learning, analisar essas frases;
\item Utilizando técnicas de Machine Learning em conjunto com bases de dados de conhecimento, criar um modelo capaz de classificar uma análise e atribuir uma ação.
\item Construir relações entre sujeito e objeto e através dessas mesmas, gerar todas(ou quase todas) as expressões regulares correspondentes.
\end{enumerate}

\section{Descrição abstrata do modelo de execução}

\hspace{11pt} Após a análise do problema em questão, foi criado um modelo de execução de um Bot.\\


\begin{align*}
DEF_{\text{Bot}} &::= Regra^* 
\end{align*}
\begin{align*}
Regra ::= antecedente & : ERL \\
\text{ reação} & : FR^* 
\end{align*}


Onde $FR$ abrevia Função de Reação e $ERL$ significa Expressão Regular Linguística. \\

Com este modelo, é possível escrever uma álgebra de Bot's que permite o uso de vários Bot's na mesma solução, por exemplo, em paralelo, em sequência, etc.

\subsection{Expressões Regulares Linguísticas}
\hspace{11pt} Estas são as expressões definidas pelo utilizador que irão ser compiladas e interpretadas pelo Bot. \\

\begin{align*}
ERL &  ::=  EL^*   \\
EL & ::= IT \quad FR^* \quad | \quad  IT2 \quad  FR^*\\
IT  &::= ( pal : ER, \quad tag : ER, \quad catch : ER ) \\
IT2 &::= (\text{``relax'', \hspace{4pt}} \quad tag : ER, \quad catch : ER) \\
FR &::= fun : ER
\end{align*}

Onde IT e IT2 significam Item e $ER$ significa Expressão Regular.\\
Os campos $pal$, $tag$ e $catch$ são expressões regulares com o significado de palavra, etiqueta e capturar, respetivamente.\\

Com estas expressões regulares, podemos contruir expressões simples, onde o Bot irá compilar em expressões regulares ordinais com todas as propriedades definidas.

\subsection{Função verifica}

\hspace{11pt} No momento em que o utilizador introduz uma frase, o Bot terá de verificar se a frase introduzida tem correspondência. É nesse momento que a função verifica atua.

$$
verifica(frase, EL) \mapsto (id \hookrightarrow val) \quad | \quad \bot
$$

Esta função aceita uma frase e uma expressão regular, e devolve um resultado de dois possíveis: um dicionário com o id e valor das expressões que se captaram; ou dá vazio quando a expressão não corresponde à frase.


\subsection{Função executa}

\hspace{11pt} Esta função é a função mostra o funcionamente do Bot.

\begin{align*}
&executa(DEF_{\text{Bot}}, str) && \\
& \quad for(antecedente, \text{reação} \in DEF_{\text{Bot}}) && \\
& \qquad v = verifica(str, antecedente) && \\
& \qquad r1 = sortear(\text{reação}) && \\
& \qquad r1(v) &&
\end{align*}


\section{Modelo V2}

\hspace{11pt} Após uma análise ao modelo anterior, foram feitas algumas alterações para simplificar a representação e implementação do modelo. Segue, então, abaixo o novo modelo linguístico:

\begin{align*}
ERL &::= EL^* \\
EL &::= EL \quad \text{``/''} \quad IT \quad ?^* \\
ER &::= IT \quad | \quad (\backslash w = IT) \\
IT &::= \$ \backslash w \quad | \quad \backslash w
\end{align*}

Onde,
\begin{align*}
ER &\leftarrow \text{Expressão Regular}\\
IT &\leftarrow \text{Item}\\
\backslash w &\leftarrow [a-zA-Z0-9]^* \\
ERL &\leftarrow \text{Expressão Regular Linguística}\\
.^* &\leftarrow \backslash w^*
\end{align*}

Explicação:
\begin{itemize}
\item $.^* / .^*$ : significa que podemos ter qualquer coisa de um tipo qualquer;
\item $. ^* / \$verbo$ : significa que podemos ter qualquer coisa de um tipo chamado $verbo$;
\item $(lugar=.^*) / \$cidades$ : significa que podemos ter qualquer coisa, que vai ser guardada numa variável chamada de $lugar$, captura de variáveis, do tipo cidades;
\item  $.^* / .^* ?$ : significa que podemos ter ou nao qualquer coisa de um tipo qualquer.\\
\end{itemize}

Estes tipos, estão definidos num mapa de subtituição que é utilizado pelo compilador do Modelo Linguístico.\\

Um exemplo de uma expressão deste modelo é:
\begin{equation}
.^* /  pronome\_geral \quad  .^* /  verbo\_geral \quad (name=.^*) / nome\_proprio\_geral \quad  ? / Fit ?
\end{equation}

Nesta expressão fazemos a correspondência de um pronome qualquer seguindo um verbo qualquer, onde de seguida capturámos um nome próprio para a variável $name$ e podemos ter ou não o sinal de pontuação $?$

\section{Arquitetura}

\hspace{11pt} 

\subsection{FreeLing}

\hspace{11pt} FreeLing é uma biblioteca escrita em C++ que analisa frases de linguagens naturais morfossintáticamente. Com esta biblioteca, é possível analisar qualquer frase e obter uma análise com grande probabilidade de estar correta, pois é utilizado um modelo de Machine Learning que foi treinado com o $Bosque$ da $Linguateca$.

Este $Bosque$ contém milhares de frases analisadas por Linguístas,  que o torna um excelente conjunto de treino para qualquer modelo de análise morfossintática.

\subsubsection{Servidor}

\hspace{11pt} Devido ao facto do modelo de análise do $FreeLing$ demorar bastante tempo a ser construído, implementou-se um pequeno servidor que tem como objetivo gerar todas as variáveis necessárias para o funcionamento do $FreeLing$ e fornecer também ao utilizador uma consola para gestão e teste rápido das funcionalidades do servidor.

\subsubsection{Cliente}

\hspace{11pt} Cada instância que precise de utilizar recursos do $FreeLing$, utilizam-se $sockets$ para se ligar ao servidor e utilizar o protocolo $JSON$ para enviar e receber informação.

\subsubsection{Wrapper}

\hspace{11pt} 

\subsection{Main}

\subsection{Ações}

\subsection{Estado}

\hspace{11pt} POR FAZER

\subsection{Base de Dados}

\subsubsection{Frases e ações}

\subsubsection{Palavras-chaves}

\subsubsection{Utilizadores}

\subsection{Configurações}

\subsection{Expressões Regulares}

\subsection{Machine Learning}

\subsection{\textit{Google Maps} Wrapper}

\subsection{\textit{Wikipedia} Wrapper}

\end{document}
